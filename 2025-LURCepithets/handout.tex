\RequirePackage{silence} % silence errors and warning I can't do anything about

% https://tex.stackexchange.com/a/108523/258167
% problem of biblatex + gb4e. Safe to ignore iff you are NOT using ibid style
% gb4e redefines \footnotetext for examples in footnotes
% biblatex catches this.
% This is probably something to resolve eventually.
% Check out https://tex.stackexchange.com/a/327725/258167 for ideas on that
\WarningFilter*{biblatex}{Patching footnotes failed.\MessageBreak Footnote detection will not work}

\documentclass{article}

\usepackage{biblatex}
\usepackage{enumitem}
\usepackage{gb4e}
\usepackage{amsmath}

\addbibresource{poster.bib}

\begin{document}
Hi. Our grand ideas of making a handout turned out to be much bigger than our actual time to make a handout out.

Here are some ramblings and other things we went through but didn't quite want to get rid of while writting the original poster:

\section{Upshot; What are epithets? (revisited)}
So, to explain this distribution, we have two options to define epithets. Either they are...
\begin{enumerate}[label=$\star$]
	\item NP/DPs accompanied by an obligatory determiner
	\item carriers of evaluative meaning
\end{enumerate}
or they
\begin{enumerate}[label=$\star$]
	\item contain pronouns ubnderlyingly, and
\end{enumerate}
\cite{aounEpithets2000,patel-groszEpithetsSyntaxsemanticsInterface2015,pottsExpressiveDimension2007,schlenkerMinimizeRestrictorsNotes2005}

\begin{exe}
	\ex {
		\textbf{John_1} ran over a man (who was) trying to give \textbf{the idiot_1} directions.
		\hfill\cite{dubinskyEpithetsAntilogophoricPronouns1998}:12
	}\label{ex:classic-epithet}
\end{exe}

What exactly are epithets? In truth, our analysis along with much of the literature focuses on basically exactly \textit{the fool/bastard/idiot} as the prototypical forms. However, our given definition covers a much broader range of usages. Further study should give a strong argument for why we choose this as the definition. Moreover, a variety of examples with varying head determiners and sematic positivity/negativity, genericity, and definiteness along with otherwise exploring what can and cannot be epithets. Finally, comparison from epithets in literary studies: ``The epithets are decorative insofar as they are neither essential to the immediate context nor modeled especially for it. Among other things, they are extremely helpful to fill out a half-verse.'' \cite{burkertEbookOrientalizingRevolution1992}

\section{Epithets are not pronouns} % Working title

\begin{exe}
	\ex {
		o = grammatical
		* = ungrammatical
		? = odd
		x = no data (e.g. failed base case)
		\textit{Donkey Anaphora:} --- I don't have a split judgement here. Do we keep?
		\begin{xlist}
			\ex {Every teacher who has a terrible student_1 fails the idiot_1.} ooooo
			\ex {Every teacher who has a terrible student_1 fails him_1.} ooooo
		\end{xlist}
	}\label{ex:donkey}

	\ex {
		1 = only meaning 1
		2 = only meaning 2
		o = both meanings
		* = neither meanings
		\textit{Paycheck Anaphora:}
		\begin{xlist}
			\ex[] {The man who sent his son to war was wiser than the man who sent him to college.} 1oo11
			\ex[\#] {The man who sent his son to war was wiser than the man who sent the idiot to college.} 1oo11
			\ex[] {
				\begin{xlist}
					\ex {Meaning 1: two different men send their own sons to different places}
					\ex {Meaning 2: two different men send the same son to different places}  maybe gay couple
				\end{xlist}
			}
		\end{xlist}
	}\label{ex:paycheck}
\end{exe}

\begin{xlist}
	\ex {
		\textbf{Topicalization:}
		\begin{xlist}
			\ex[] {Himself_1, John_1 tripped over yesterday.} *x*xx
			\ex[] {Him_1, John_1 tripped over yesterday.} *x*xx
			\ex[] {The idiot_1, John_1 tripped over yesterday.}  oxoxx
		\end{xlist}
	}
	\ex {
		\textbf{Relative Clause:}
		\begin{xlist}
			\ex[] {It was himself_1 that John_1 tripped over yesterday.} ooo?o
			\ex[] {It was him_1 that John_1 tripped over yesterday.} ***o*
			\ex[] {It was the idiot_1 that John_1 tripped over yesterday.} ***o*
		\end{xlist}
	}
	\ex {
		\textbf{Wh-Questions:}
		\begin{xlist}
			\ex[] {John_1 showed Mary a terrible picture of the idiot_1.} o?o*o
			\ex[] {Which terrible picture of himself_1 did John_1 show Mary.} oxooo
			\ex[?] {Which terrible picture of him_1 did John_1 show Mary.} *xoo*
			\ex[?] {Which terrible picture of the idiot_1 did John_1 show Mary.} ?x?*o
			\ex[?] {Which terrible picture of John_1 did he_1 show Mary.} *x*o*
			\ex[] {Which terrible picture of John_1 did the idiot_1 show Mary.} *x**o
			\ex[*] {Who_1 did he_1 trip over?} oxooo
			\ex[] {Who_1 did the idiot_1 trip over?} oxooo
		\end{xlist}
	}
\end{xlist}

\section{Epithets can be Bound Variable Anaphora}
TODO: 'all the idiots'; quantificationally introduced nouns are generally not considered anaphoric; are these even evaluative?
Each student got help from John. Each idiot subsequently ignored him.
The teacher graded the squib that each idiot wrote.
Context A: Each student wrote their own terrible squib
*Context B: Each student collectively wrote a terrible group squib
The teacher graded each squib that some idiot wrote.
Context A: A single student writes terrible squibs
Context B: each terrible squib is written be a different student

As seen below, epithets can act as bound variables in quantificational and conjoined phrases:
\begin{exe}
	\ex {\textbf{\(\boldsymbol{\forall}\):}
		Each driver_1 ran over the man who was trying to give the idiot_1 directions.\\
		(Each driver)(\(\lambda\) x.x ran over the man who was trying to give x directions)
	}
	\ex {\textbf{and:}
		Context: John and Tom are bad students. A different teacher helped each of them.\\
		John_1 hit the teacher who helped the idiot_1 and Tom_2 did too.\\
		(John)(\(\lambda\) x . x hit the teacher who helped x) AND (Tom)(\(\lambda\) x . x hit the teacher who helped x)
	}
	TODO: quantificational introduction and information structure
	\ex {After the students_1 got help from John, the idiots_1 subsequently ignored him.}
	\ex {
		After the students_1 got help from John, all the students_1/idiots_1 ignored him on the test.
	}
	\ex {
		After John helped the students, they/all the idiots/all the students ignored him.
	}
\end{exe}
\end{document}
